% vim: tw=80 noai
\documentclass[normaltoc,capchap,capsec,times]{abnt}
\usepackage[utf8]{inputenc}
\usepackage[T1]{fontenc}
\usepackage[brazil]{babel}
\usepackage[alf]{abntcite}
\usepackage[ordem=alf]{tabela-simbolos}
\usepackage{url}
\usepackage{graphicx}
\usepackage{listings}
\usepackage{verbatim}
\usepackage{subfigure}
\usepackage{multicol}
\usepackage{framed}
\usepackage{multirow}
% O comando abaixo define o diretorio onde devem ser colocadas as imagens.
% Neste caso o diretorio é ./imagens/
\graphicspath{ {./imagens/} }
\def\lstlistingname{Listagem}

%%%%%%%%%%%%%%%%%%%%%%%%%%%%%%%%%%%%%%%%%%%%%%%%%%%%
% Dados da monografia
%%%%%%%%%%%%%%%%%%%%%%%%%%%%%%%%%%%%%%%%%%%%%%%%%%%%

\newcommand{\meunome}{José Lucas dos Santos Borges}
\newcommand{\meutitulo}{Um método sensível ao contexto para predição de momentos oportunos para interrupção de motoristas}
\newcommand{\meusubtitulo}{}
\newcommand{\meuano}{Outubro/2016}
\newcommand{\meuorientador}{Orientador: \profa\ Vaninha Vieira dos Santos}
\newcommand{\meucoorientador}{Co-Orientador: Alberto Vianna Dias da Silva}

%%%%%%%%%%%%%%%%%%%%%%%%%%%%%%%%%%%%%%%%%%%%%%%%%%%%

%% O comando \obs aqui definido permite que o autor faca anotacoes na
%% monografia que aparecem no PDF gerado. Para ativar o comando, descomente
%% a primeira linha e comente a segunda.
%% Exemplo de uso: \obs{Preciso melhorar este parágrafo...}

%\newcommand{\obs}[1]{\underline{\textbf{OBSERVAÇÃO}}: \emph{#1}}
\newcommand{\obs}[1]{}

\def\ordfem{\mbox{\raise .35em \hbox{\underline{\scriptsize a}\ }}}
\def\ordmasc{\mbox{\raise .35em \hbox{\underline{\scriptsize o}\ }}}
\def\profa{Prof\ordfem.}

%%%%%%%%%%%%%%%%%%%%%%%%%%%%%%%%%%%%%%%%%%%%%%%%%%%%

\begin{document}

\input{capa.tex}

%\folhaderosto
% o comando acima foi comentado para não criar uma folha de
% rosto redundante, já que ela feita 'manualmente' abaixo

\begin{titlepage}
 \vfill
 \begin{center}
   {\large \uppercase{ \bf{ \meunome\ } } } \\[7cm]
   {\Huge \uppercase{ \bf{ \meutitulo\ } } }\\[1cm]
   \vfill
   \hspace{.45\textwidth} % posicionando a minipage
   \begin{minipage}{.5\textwidth}
     \begin{espacosimples}
       \bf{
	Anteprojeto da Monografia apresentada ao curso de graduação em Sistemas de Informação,
	Departamento de Ciência da Computação da Universidade Federal da
	Bahia, como requisito parcial para obtenção do grau de Bacharel em Sistemas de Informação. \\
       }
     \end{espacosimples}
     \begin{espacosimples}
       \meuorientador
       \newline
       \meucoorientador
     \end{espacosimples}
   \end{minipage}
   \vfill
   Salvador - Bahia \\
   \meuano
 \end{center}
\end{titlepage}


%% As listas a seguir sao opcionais:
%\listadetabelas
\listadesiglas
%\listadesimbolos

\sumario

% O conteudo da monografia esta' nos seguintes arquivos:
\chapter{Introdução}
\label{introducao}


\section{Estrutura da Dissertação}
\label{estrutura}

\input{interrupcao.tex}
\sigla{GPS}{Sistema de Posicionamento Global}
\sigla{UX}{User Experience}
\sigla{OBD}{On-Board Diagnostic}

\chapter{Revisão da Literatura}
\label{revisao-lit}
Este capítulo aborda os principais fundamentos teóricos envolvidos na notificação oportuna de motoristas,
tema central deste trabalho, e traz uma revisão dos trabalhos relacionados com o tema. São apresentados
conceitos sobre interrupção, notificação e contexto. Ao final são apresentados alguns trabalhos correlatos.

\section{Interrupção}
\label{interrupcao}

Segundo \cite{ferreira2004novo}, interrupção é aquilo que faz parar uma ação ou um estado; o ato de cortar a continuidade de
algo. A interrupção durante a execução de uma tarefa pode ter vários efeitos adversos. \cite{lewin1927untersuchungen} diz
que pessoas lembram melhor dos detalhes de tarefas que não foram interrompidas. \cite{zijlstra1999temporal} conclui que
pessoas cometem mais erros em tarefas após uma interrupção. \cite{gillie1989makes} afirma que as pessoas executam tarefas
mais vagarosamente após uma interrupção, se comparado com a performance pré-interrupção. Diante destas evidências, pode-se
afirmar que interrupções durante uma tarefa são bastante nocivas para a execução da mesma.

\subsection{Interrupção de Motoristas}
\label{interrupcao-motoristas}

Trabalhos anteriores estudaram os efeitos da execução de tarefas concorrentes com a direção. \cite{monk2004recovering} cita
que há diversos efeitos adversos ao executar tarefas cognitivas complexas durante a direção, como atraso na resposta a
acontecimentos repentinos, desatenção a informações sinalizadas, diminuição do controle do veículo, estreitamento do campo
de visão e mudanças de comportamento de frenagem e direção.

Além disso, vários problemas na execução de uma tarefa após uma interrupção, como os citados na seção \ref{interrupcao}, afetam o
motorista durante a direção de um veículo:

\begin{itemize}
  \item Ao não lembrar de detalhes do que estava fazendo antes da interrupção, o motorista pode esquecer de informações
  apontadas pela sinalização de trânsito;
  \item Ao cometer erros após uma interrupção, o motorista põe em risco a si mesmo e a seus pares, podendo causar acidentes
  de trânsito;
  \item Ao reagir mais vagarosamente após uma interrupção, o motorista fica vulnerável a ameaças externas que exijam de sua
  capacidade reativa;
\end{itemize}



Estima-se que o uso de celular durante o ato de dirigir um veículo aumenta o risco de acidentes em 38\% \cite{laberge2001wireless}.
Em paralelo a este fato, \cite{stothart2015attentional} afirma que apenas o ato de receber uma notificação, mesmo que ela não seja
atendida, distrai o motorista tanto quanto receber uma ligação no celular ou responder uma mensagem de texto. Ante este fato, é
necessário estudar as notificações e sua capacidade de interromper tarefas.

\section{Notificações e seu caráter interruptivo}
\label{notificacao}

TODO - Caráter interruptivo da notificação. Mencionar que no momento da chegada, alertas são realizados, a atenção do
usuário é interrompida e ele foca no dispositivo.

TODO - Gancho para seção de contexto. Meio que simples, pois é só citar que para saber o momento certo de notificar
é importante saber a situação na qual o usuário se encontra e o ambiente a seu redor, que dá pra fazer utilizando contexto.

Notificação pode ser definida como um sinal visual, audível ou táctil, gerado por uma aplicação
ou serviço e que passa informação para um usuário que está fora de seu foco de atenção \cite{iqbal2010notifications}.
Em dispositivos móveis, notificações geralmente são enviadas instantaneamente no momento em que ocorre alguma atividade que pode ser relevante
para o usuário quando a aplicação não está aberta, ex: um email novo, uma mensagem de texto que acaba de chegar ou um
novo comentário em suas redes sociais. Em alguns casos o usuário toma ações imediatas após a chegada da notificação,
enquanto em outros ela é simplesmente ignorada. Essas ações dependem da importância da notificação e do contexto do
usuário \cite{sahami2014large}.

Em dispositivos móveis uma notificação é uma mensagem que pode ser exibida ao usuário fora da interface normal de um aplicativo.
Quando o aplicativo emite uma notificação, ela primeiro aparece como um ícone na área de notificação. Para ver os detalhes da
notificação, o usuário abre a gaveta de notificação \cite{notificationDrawer}. A figura \ref{notification-drawer} mostra
o layout da gaveta de notificações no Android. Cada retângulo branco preenchido com ícones e texto representa uma notificação
diferente. O horário de chegada aparece no canto direito de cada notificação.

Notificações semelhantes geralmente são agrupadas para que não sejam exibidas inúmeras notificações com conteúdo parecido.
Pode-se perceber na figura \ref{notification-drawer} que a maioria das notificações possui um indicativo de informações agrupadas
ao invés de múltiplas notificações. Ex: O texto "\textit{3 new messages}" que aponta que existem 3 novas mensagens, ao invés de 3
notificações apontando 1 nova mensagem cada.

\begin{figure}[h]
\centering
\includegraphics[width=0.3\textwidth]{notification_drawer}
\caption{Exemplo de notificações no Android \cite{notificationDrawer}}
\label{notification-drawer}
\end{figure}

A área de notificação e a gaveta de notificação mostradas na figura \ref{notification-drawer} são áreas controladas pelo
sistema operacional e que o usuário pode visualizar a qualquer momento.

A maioria das notificações possui pelo menos uma ação atrelada a si. Uma ação permite que os usuários passem
diretamente da notificação para uma tela específica do aplicativo, onde podem visualizar um ou mais eventos ou realizar
outros trabalhos.

\section{Contexto}
\label{contexto}

Em 1991 é cunhado o termo "Computação Ubíqua", que se refere ao caráter invisível da integração de dispositivos
computacionais diversos e da adaptação dos mesmos à necessidade do usuário no momento \cite{weiser1991computer}.
Um elemento bastante importante para a Computação Ubíqua é o estudo do contexto.

Contexto é definido por \cite{dey2001understanding} como "Qualquer informação que pode ser utilizada para
caracterizar a situação de entidades (ex: um usuário, lugar ou objeto) e que é considerada relevante para
a interação entre um usuário e uma aplicação, incluindo o próprio usuário e aplicação". Esta definição é
a mais utilizada na área. Alguns exemplos de elementos de contexto são
localização do usuário, identidade do usuário e tempo \cite{ryan1999enhanced}.

Diversos sensores podem ser utilizados para determinar informações sobre o contexto do usuário. Alguns exemplos são
sensores de localização (GPS), sensores de luz, acelerômetro e giroscópio.

Informações de contexto são importantes para definir o estado atual do usuário e do ambiente onde ele está inserido,
mas somente isto é insuficiente. Para utilizar estas informações satisfatoriamente é ideal que se construa um sistema
adaptativo e que supra as necessidades do usuário em tempo real utilizando as informações de contexto. Resumindo,
um sistema sensível ao contexto.

Sistemas sensíveis ao contexto são capazes de adaptar suas operações ao contexto atual, sem intervenção
explícita do usuário e têm como objetivo aumentar sua usabilidade e efetividade levando em conta elementos
de contexto \cite{baldauf2007survey}.  Um sistema é sensível ao contexto se ele utiliza contexto para prover
informações relevantes e/ou serviços para usuários, sendo que a relevância depende das tarefas do usuário
\cite{abowd1999towards}.

Este trabalho propõe a construção de uma ferramenta que utilize contexto para a detecção de momentos oportunos para
o envio de notificações a motoristas, utilizando apenas sensores de um smartphone comum.

\section{Trabalhos Relacionados}
\label{trabalhos-relacionados}

TODO - Criticar os trabalhos

\cite{park2016integrated} desenvolveu um sistema integrado com direção veicular para conduzir estudos de experiência
do usuário (UX). Este sistema coleta dados do motorista e do veículo, analiza-os e provê um feedback visual baseado
nas informações que o motorista precisa no momento. São utilizados sensores vestíveis para aquisição de dados
fisiológicos e de movimentação do motorista, um dispositivo OBD para obtenção de dados veiculares, um GPS veicular e
as informações são exibidas em um aplicativo Android.

\cite{eren2012estimating} apresenta um sistema para estimar o comportamento na direção de motoristas como seguro ou
não seguro, utilizando algoritmo de detecção de melhor caminho e classificação Bayesiana. Este sistema utiliza
apenas sensores de smartphone (acelerômetro, giroscópio e magnetômetro) para adquirir dados de contexto e obteve
resultados corretos em 14 casos de 15. \cite{johnson2011driving} apresenta um sistema semelhante também utilizando
apenas sensores de smartphone e obteve cerca de 97\% de corretude. Estes resultados demonstram que é possível construir
um mecanismo para predição de comportamento veicular utilizando apenas sensores do smartphone e podem contribuir
para este trabalho.

Trabalhos anteriores também corroboram a tese de que é possível utilizar sensores para prever a interruptibilidade
humana em um dado momento. \cite{fogarty2005predicting} construiu um modelo que utiliza sensores de áudio e vídeo
para estimar a interruptibilidade de trabalhadores de escritório em um dado momento. \cite{hudson2003predicting}
construiu um modelo semelhante utilizando os mesmos sensores e obteve uma média de 75\% de acurácia na predição de
interruptibilidade de trabalhadores de escritório.

\cite{kim2015sensors} desenvolveu um classificador utilizando aprendizagem de máquina para detectar a interruptibilidade
de motoristas em um momento. Para a coleta de dados foram utilizados um sensor fisiológico vestível, quatro sensores
de movimento também vestíveis, câmeras direcionadas ao tráfego e ao motorista e um dispositivo OBD. O classificador
construído obteve 94\% de precisão na detecção de momentos oportunos para interrupção de motoristas. O presente
trabalho visa construir um classificador semelhante, com o diferencial de utilizar apenas sensores de smartphone.

\chapter{Proposta}
\label{proposta}
Este capítulo aborda alguns aspectos da proposta para o projeto final. As seguintes seções são apresentadas:
Objetivo, Metodologia, Resultados Esperados e Cronograma.

\section{Objetivo}
\label{objetivo}
O objetivo desta pesquisa utilizar elementos de contexto para desenvolver um mecanismo que faça predições de
momentos oportunos para interrupção de motoristas, usando apenas sensores de smartphone.

\subsection{Objetivos Específicos}
\label{objetivos-esp}
\begin{itemize}
  \item Investigar elementos de contexto que indiquem que é possível interromper um motorista em um dado momento;
  \item Investigar e aplicar técnicas que podem ser usadas para a predição de interrupções de motoristas;
  \item Desenvolver uma solução que implemente os pontos anteriores de forma satisfatória;
  \item Implementar a solucão no aplicativo Meu Possante \cite{meupossante};
  \item Avaliar o desempenho do mecanismo de interrupções;
\end{itemize}

\section{Metodologia}
\label{metodologia}
\begin{itemize}
  \item Serão identificados momentos oportunos para interrupção de motoristas a partir da leitura de artigos
    relacionados;
  \item Será feita a concepção e definição do escopo do artefato a ser desenvolvido, de forma que atenda a
    todos os objetivos específicos;
  \item A solução será projetada na forma de uma biblioteca que prediz momentos oportunos para a interrupção
    de motoristas;
  \item A biblioteca será desenvolvida para dispositivos do sistema operacional Android;
  \item A biblioteca será incluída no aplicativo Meu Possante \cite{meupossante} e serão feitos testes que analisem se ela funciona
    corretamente;
  \item A solução será testada e avaliada;
\end{itemize}

\section{Resultados Esperados}
\label{resultados}

Ao final deste trabalho espera-se ter um mecanismo que faça a predição automática de momentos oportunos para
interrupção de motoristas, implementado na forma de uma biblioteca que poderá ser utilizada por qualquer aplicativo
do sistema operacional Android.

\section{Cronograma}
\label{cronograma}

\begin{table}[h]
\centering
\resizebox{\textwidth}{!}{
\begin{tabular}{|r|c|c|c|c|c|c|c|c|c|}
\hline
\multicolumn{1}{|c|}{\multirow{2}{*}{Atividade}}                                                        & \multicolumn{2}{c|}{Dezembro} & \multicolumn{2}{c|}{Janeiro} & \multicolumn{2}{c|}{Fevereiro} & \multicolumn{2}{c|}{Março} & Abril     \\ \cline{2-10}
\multicolumn{1}{|c|}{}                                                                                  & 1ª quinz.     & 2ª quinz.     & 1ª quinz.     & 2ª quinz.    & 1ª quinz.      & 2ª quinz.     & 1ª quinz.    & 2ª quinz.   & 1ª quinz. \\ \hline
\begin{tabular}[c]{@{}r@{}}Pesquisa de momentos oportunos\\ para interrupção de motoristas\end{tabular} & X             & X             &               &              &                &               &              &             &           \\ \hline
\begin{tabular}[c]{@{}r@{}}Elaboração da arquitetura\\ da biblioteca\end{tabular}                       &               & X             & X             &              &                &               &              &             &           \\ \hline
\begin{tabular}[c]{@{}r@{}}Testes com sensores\\ que serão utilizados\end{tabular}                      &               &               & X             &              &                &               &              &             &           \\ \hline
Desenvolvimento da biblioteca                                                                           &               &               & X             & X            & X              & X             &              &             &           \\ \hline
\begin{tabular}[c]{@{}r@{}}Implementação no aplicativo\\ Meu Possante\end{tabular}                      &               &               &               &              &                & X             &              &             &           \\ \hline
Testes no aplicativo                                                                                    &               &               &               &              &                & X             &              &             &           \\ \hline
Avaliação da solução                                                                                    &               &               &               &              &                &               & X            &             &           \\ \hline
Escrita da monografia                                                                                   &               & X             & X             & X            & X              & X             & X            & X           &           \\ \hline
Entrega do projeto                                                                                      &               &               &               &              &                &               &              & X           &           \\ \hline
Defesa do PF II                                                                                         &               &               &               &              &                &               &              &             & X         \\ \hline
\end{tabular}
}
\end{table}


\bibliography{tcc}

\end{document}
