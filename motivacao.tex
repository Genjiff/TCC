\chapter{Motivação}
\label{motivacao}
Dirigir um veículo pode ser considerado uma tarefa complexa, já que envolve diversas pequenas ações coordenadas e
que dependem tanto de fatores externos quanto internos. Um motorista deve saber se adaptar a diferentes situações,
agir apropriadamente perante cada uma delas e tomar medidas rápidas em caso de emergência.

Neste contexto, a interrupção inoportuna é um perigo que pode ser fatal, pois demanda atenção do motorista em um momento
onde a distração pode levar a um acidente. Um fato que corrobora isto é o de que pessoas levam até 27\% mais tempo para
concluir uma tarefa quando são interrompidas e tendem a cometer o dobro de erros \cite{bailey2006need}. Exemplos de interrupções que podem
ocorrer para motoristas são barulhos repentinos, conversas com passageiros e notificações em seu celular.

Diariamente, pessoas que utilizam qualquer tipo de dispositivo móvel recebem dezenas de notificações \cite{pielot2014situ}. Tais notificações
demandam a atenção do usuário e podem acabar interrompendo uma tarefa que está sendo executada no momento, dado que a
atenção humana é um recurso finito \cite{simon1971designing}.

Interrupções de motoristas já são um grave problema. 10\% dos acidentes com morte nos Estados Unidos em 2014 tem relação
com algum tipo de distração e 13\% deste tipo de distrações estão relacionadas com o uso de celulares e smartphones \cite{distracted2014}.
Diante disto, prevenir este tipo de distração relacionada a dispositivos móveis torna-se uma necessidade.

A utilização de dispositivos móveis enquanto se está dirigindo, apesar de ser proibido no Brasil é liberado em alguns
outros países, como Estados Unidos (em alguns estados como Flórida e Colorado \cite{cellphoneuse} \cite{distracteddriving} e Suécia \cite{swedendrive}.

Identificando padrões nos dados contextuais que indiquem momentos oportunos para interromper motoristas, é possível
construir uma ferramenta que os notifique apropriadamente a depender do contexto que eles se encontram e que possa ajudar
na diminuição das estatísticas de morte causadas por distração.
