\chapter{Fundamentação}
\label{fundamentacao}
Este capítulo aborda os principais fundamentos teóricos envolvidos na notificação oportuna de motoristas,
tema central deste trabalho. São apresentados conceitos sobre contexto, interrupção e notificação.
Ao final são apresentados alguns trabalhos correlatos.

\section{Contexto}
\label{contexto}
Em 1991, \cite{weiser1991computer} cunha o termo "Computação Ubíqua", que se refere ao caráter invisível da
integração de dispositivos computacionais diversos e da adaptação dos mesmos à necessidade do usuário no momento.
Um elemento bastante importante para a Computação Ubíqua é o estudo do contexto.

Contexto é definido por \cite{dey2001understanding} como "Qualquer informação que pode ser utilizada para
caracterizar a situação de entidades (ex: um usuário, lugar ou objeto) e que é considerada relevante para
a interação entre um usuário e uma aplicação, incluindo o próprio usuário e aplicação". Esta definição é
a mais utilizada na área e provavelmente a mais aceita. Alguns exemplos de elementos de contexto são
localização do usuário, ambiente, identidade do usuário e tempo \cite{ryan1999enhanced}.

Informações de contexto são importantes para definir o estado atual do usuário e do ambiente onde ele está
inserido, mas somente isto é insuficiente. Para utilizar estas informações satisfatoriamente, é ideal que
se construa um sistema adaptativo e que supra as necessidades do usuário em tempo real utilizando as
informações de contexto. Resumindo, um sistema sensível ao contexto.

Sistemas sensíveis ao contexto são capazes de adaptar suas operações ao contexto atual, sem intervenção
explícita do usuário e têm como objetivo aumentar sua usabilidade e efetividade levando em conta elementos
de contexto \cite{baldauf2007survey}. Já \cite{abowd1999towards} define que um sistema é sensível ao contexto
se ele utiliza contexto para prover informações relevantes e/ou serviços para usuários, sendo que a relevância
depende das tarefas do usuário.



\section{Notificação}
\label{notificacao}

\section{Interrupção}
\label{interrupcao}

\subsection{Interupção de Motoristas}
\label{interrupcao-motoristas}
