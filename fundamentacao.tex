\chapter{Fundamentação}
\label{fundamentacao}
Este capítulo aborda os principais fundamentos teóricos envolvidos na notificação oportuna de motoristas,
tema central deste trabalho. São apresentados conceitos sobre contexto, interrupção e notificação.
Ao final são apresentados alguns trabalhos correlatos.

\section{Contexto}
\label{contexto}
Em 1991, \cite{weiser1991computer} cunha o termo "Computação Ubíqua", que se refere ao caráter invisível da
integração de dispositivos computacionais diversos e da adaptação dos mesmos à necessidade do usuário no momento.
Um elemento bastante importante para a Computação Ubíqua é o estudo do contexto.

Contexto é definido por \cite{dey2001understanding} como "Qualquer informação que pode ser utilizada para
caracterizar a situação de entidades (ex: um usuário, lugar ou objeto) e que é considerada relevante para
a interação entre um usuário e uma aplicação, incluindo o próprio usuário e aplicação". Esta definição é
a mais utilizada na área e provavelmente a mais aceita.

\cite{ryan1999enhanced} apresenta exemplos 

Detectar características e mudanças na situação dos usuários e do ambiente tem se tornado muito

\section{Notificação}
\label{notificacao}

\section{Interrupção}
\label{interrupcao}
