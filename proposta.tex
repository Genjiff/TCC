\chapter{Proposta}
\label{proposta}
Este capítulo aborda alguns aspectos da proposta para o projeto final. As seguintes seções são apresentadas:
Objetivo, Metodologia, Resultados Esperados e Cronograma.

\section{Objetivo}
\label{objetivo}
O objetivo desta pesquisa é desenvolver um mecanismo que faça predições de momentos oportunos para interrupção de
motoristas, utilizando elementos de contexto.

\subsection{Objetivos Específicos}
\label{objetivos-esp}
\begin{itemize}
  \item Investigar elementos de contexto que indiquem que é possível interromper um motorista em um dado momento;
  \item Investigar e aplicar técnicas que podem ser usadas para a predição de interrupções de motoristas;
  \item Desenvolver uma solução que implemente os pontos anteriores de forma satisfatória;
  \item Implementar a solucão no aplicativo Meu Possante \cite{meupossante};
  \item Avaliar o desempenho do mecanismo de interrupções;
\end{itemize}

\section{Metodologia}
\label{metodologia}
\begin{itemize}
  \item Serão identificados momentos oportunos para interrupção de motoristas a partir da leitura de artigos
    relacionados;
  \item Será feita a concepção e definição do escopo do artefato a ser desenvolvido, de forma que atenda a
    todos os objetivos específicos;
  \item A solução será projetada na forma de uma biblioteca que prediz momentos oportunos para a interrupção
    de motoristas;
  \item A biblioteca será desenvolvida para dispositivos do sistema operacional Android;
  \item A biblioteca será incluída no aplicativo Meu Possante \cite{meupossante} e serão feitos testes que analisem se ela funciona
    corretamente;
  \item A solução será testada e avaliada;
\end{itemize}

\section{Resultados Esperados}
\label{resultados}

Ao final deste trabalho espera-se ter um mecanismo que faça a predição automática de momentos oportunos para
interrupção de motoristas, implementado na forma de uma biblioteca que poderá ser utilizada por qualquer aplicativo
do sistema operacional Android.

\section{Cronograma}
\label{cronograma}

\begin{table}[h]
\centering
\resizebox{\textwidth}{!}{
\begin{tabular}{|r|c|c|c|c|c|c|c|c|c|}
\hline
\multicolumn{1}{|c|}{\multirow{2}{*}{Atividade}}                                                        & \multicolumn{2}{c|}{Dezembro} & \multicolumn{2}{c|}{Janeiro} & \multicolumn{2}{c|}{Fevereiro} & \multicolumn{2}{c|}{Março} & Abril     \\ \cline{2-10}
\multicolumn{1}{|c|}{}                                                                                  & 1ª quinz.     & 2ª quinz.     & 1ª quinz.     & 2ª quinz.    & 1ª quinz.      & 2ª quinz.     & 1ª quinz.    & 2ª quinz.   & 1ª quinz. \\ \hline
\begin{tabular}[c]{@{}r@{}}Pesquisa de momentos oportunos\\ para interrupção de motoristas\end{tabular} & X             & X             &               &              &                &               &              &             &           \\ \hline
\begin{tabular}[c]{@{}r@{}}Elaboração da arquitetura\\ da biblioteca\end{tabular}                       &               & X             & X             &              &                &               &              &             &           \\ \hline
\begin{tabular}[c]{@{}r@{}}Testes com sensores\\ que serão utilizados\end{tabular}                      &               &               & X             &              &                &               &              &             &           \\ \hline
Desenvolvimento da biblioteca                                                                           &               &               & X             & X            & X              & X             &              &             &           \\ \hline
\begin{tabular}[c]{@{}r@{}}Implementação no aplicativo\\ Meu Possante\end{tabular}                      &               &               &               &              &                & X             &              &             &           \\ \hline
Testes no aplicativo                                                                                    &               &               &               &              &                & X             &              &             &           \\ \hline
Avaliação da solução                                                                                    &               &               &               &              &                &               & X            &             &           \\ \hline
Escrita da monografia                                                                                   &               & X             & X             & X            & X              & X             & X            & X           &           \\ \hline
Entrega do projeto                                                                                      &               &               &               &              &                &               &              & X           &           \\ \hline
Defesa do PF II                                                                                         &               &               &               &              &                &               &              &             & X         \\ \hline
\end{tabular}
}
\end{table}
